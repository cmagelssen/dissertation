\textbf{Bakgrunn og mål.} I løpet av de siste tiårene har kognitiv vitenskap gjort store fremskritt i forståelsen av mekanismene som underbygger ferdighetslæring. Denne kunnskapen har inspirert tanker om hvordan man kan oppnå bedre læring ved å utnytte disse mekanismene mer effektivt. Så langt har de fleste studier fokusert på enkle, laboratoriebaserte oppgaver. Dette reiser spørsmål om i hvilken grad disse læringsprinsippene kan generaliseres til reelle læringskontekster, som i idrett, utdanning eller rehabilitering. På bakgrunn av dette er det et stort behov for studier med større økologisk validitet for å avgjøre om disse prinsippene kan forbedre viktige livsfunksjoner eller ferdigheter som er viktig for individet. Det overordnede målet med denne doktorgradsavhandlingen har derfor vært å tette dette gapet ved å bruke alpint som en testplattform for å studere ferdighetslæring hos gode idrettsutøvere.

For å oppnå dette målet, har denne doktorgradsavhandlingen brukt en tverrfaglig tilnærming forankret i både mekanikk og psykologi. Denne tverrfaglige tilnærmingen var nødvendig for å lære om strategier som kan forbedre prestasjonen til gode alpinister, men også for å studere om effektive læringsstrategier kan forbedre læringen og utførelsen av disse strategiene. Fra et mekanisk perspektiv spurte jeg hva som er de mest effektive strategiene for å kjøre raskt på flater i slalåm (forskningsmål I; Artikkel III og tilleggsanalyse i denne avhandlingen), og hva som er den kinematiske signaturen som gjør en av disse strategiene så effektiv (forskningsmål II; Artikkel II). Fra et psykologisk perspektiv spurte jeg om alpinister kan lære bedre ved å bruke kontekstuell interferens for å trene læringsoppgaver (forskningsmål III; Artikkel I), og om forsterkendelæring er et mer effektivt læringssignal for instruksjon og tilbakemelding enn tradisjonell instruksjonsbasert læring (forskningsmål IV; Artikkel III).

\textbf{Metoder.} Denne avhandlingen er basert på to hovedstudier. I Studie I testet vi om trening med høy grad av kontekstuell interferens kan forbedre læring hos 66 dyktige alpinister. For å teste dette kontrastet vi to læringsgrupper som lærte å bruke "strekk" (eller pumping) strategien på flater i slalåm: én læringsgruppe (sammenflettet læringsgruppe) som trente strategien i tre slalåmløyper hver dag i tilfeldig rekkefølge (alpinistene gjentok aldri mer enn to påfølgende repetisjoner i samme slalåmløype), og én læringsgruppe (blokkert læringsgruppe) som trente samme strategi i en slalåmløype per dag (motvektet rekkefølge på slalåmløypene mellom alpinistene). Eksperimentet strakte seg over åtte dager med en baseline-test på dag én, etterfulgt av tre dager med trening, og en retensjonstest tre dager etter den siste treningsdagen. Effekten av læringsgruppene på læring og prestasjon er rapportert i Artikkel I.  For tre av skilagene (18 alpinister) som deltok i denne studien, har vi i tillegg gjort opptak av alpinistenes posisjoner i slalåmløypa ved hjelp av et lokalt posisjoneringssystem under baseline- og retensjonstestene. Hensikten med dette var å analysere endringer i kinematikk gjennom intervensjonen for å bedre forstå den kinematiske signaturen til "strekk" (pumping) strategien. Resultatene av denne analysen er rapportert i Artikkel II.

I Studie II testet vi hvilket læringssignal for instruksjon og tilbakemelding som er mest effektivt for å lære valg av gode strategier for måloppnåelse. Dette læringseksperimentet involverte 98 gode alpinister der vi sammenlignet tre læringsgrupper og deres læring når det gjelder å gjøre effektive valg av strategier for å kjøre raskere på flatene. I den veiledede (fri valg) læringsgruppen ble alpinistene fortalt av sine trenere hvilken strategi de skulle bruke, som var enten den de mente var den beste eller mest passende for alpinisten. Denne læringsgruppen representerer den konvensjonelle læringsstrategien. Vi komplementerte denne læringsgruppen med en veiledet (målferdighet) læringsgruppe, hvor trenere instruerte alpinistene til å velge den strategien vi i forkant hadde definert som den teoretisk beste strategien basert på mekanikk og feltobservasjoner av elitealpinister. Siden alpinistene ble instruert til å bruke den teoretisk optimale strategien, tjente denne læringsgruppen som en referanse for den øvre grensen for prestasjon som er oppnåelig gjennom optimale strategivalg. I begge disse veiledede læringsgruppene så trenerne alpinistenes tider, men kommuniserte ikke tidene til alpinistene. Vi sammenlignet disse to læringsgruppene med en forsterkningslæringsgruppe, hvor alpinistene lærte verdiene av strategiene ved å prøve dem ut og å lære gjennom evalueringer fra tidene i stedet for fra en trener. Effekten av læringsgruppene på læring for å gjøre gode strategivalg for prestasjon er rapportert i Artikkel III. I denne studien rapporterte vi også analyser som sammenlignet bevegelsesstrategiene, men denne analysen er overskygget av hovedfunnene i studien og ble derfor plassert i en tilleggstabell. Derfor har jeg i denne avhandlingen reanalysert disse dataene og rapportert funnene i hovedteksten.

\textbf{Resultater.} For å begynne med de mekaniske målene for denne avhandlingen, fra Studie II (Artikkel III og tillegganalysen i denne avhandlingen), fant vi at alpinister i gjennomsnitt oppnådde de raskeste tidene ved å bruke “strekk med skyv skiene framover”-strategien, etterfulgt av ”strekk”-(pumping), ”skyv skiene framover”- og ”stå imot”-strategiene på flaten. Bemerkelsesverdig var “strekk med skyv skiene framover”-strategien bare marginalt bedre enn “strekk”-strategien (pumping), som er enklere og var nesten like effektiv på egen hånd. Fra Studie I (Paper II), fant vi at alpinistenes fartsprofil endret seg betydelig etter treningsintervensjonen på "strekk"-strategien (pumping). Etter intervensjonen nådde alpinistenes høyere fart rett etter portpasseringen. Farten fortsatte å stige til midt mellom portene, før den falt rett før neste port. Vi observerte også en trend der alpinistenes veilengde økte fra port til port, selv om dette varierte betydelig fra sving til sving. Når det gjelder de psykologiske målene for avhandlingen, fra Studie I (Paper I), fant vi ikke evidens for at den sammenflettede læringsgruppen presterte dårligere under innlæring eller forbedret retensjon, sammenlignet med den blokkerte læringsgruppen. Dette indikerer at det ikke var overbevisende evidens for den kontekstuelle interferenseffekten. På den annen side, fra Studie II (Paper III), fant vi at læring av strategier gjennom forsterkningslæring ga et mer effektivt læringssignal enn konvensjonell instruksjonsbasert læring. Interessant nok, presterte forsterkningslæringsgruppen også deskriptivt bedre enn den veiledede (målferdighet) læringsgruppen, som var ment å representere den øvre grensen for prestasjon som er oppnåelig gjennom optimale strategivalg.

\textbf{Konklusjon. }Denne avhandlingen viser at strategiene alpinister velger kan ha en avgjørende innflytelse på deres prestasjon i slalåm. Vi fant ikke overbevisende evidens for at trening med større kontekstuell interferens (sammenflettet læring) økte ferdighetslæring. Imidlertid fant vi at forsterkningslæring var et effektivt læringssignal for å trene utøvere til å gjøre gode strategivalg. Basert på disse funnene bør trenere utvikle ulike strategier og la utøvere lære deres verdier gjennom evaluering, noe som til slutt gjør utøverne i stand til å velge det beste alternativet. Totalt sett fremhever denne avhandlingen at en tverrfaglig tilnærming kan være en effektiv strategi for å studere ferdighetslæring hos eliteutøvere, og det kan representere en lovende tilnærming for fremtidig forskning.


\setlength{\parindent}{15pt} % Standard innrykk (kan justeres etter behov)
\setlength{\parskip}{0pt}    % Standard avstand mellom avsnitt (kan justeres etter behov)