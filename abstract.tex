\setlength{\parindent}{0pt} % Fjerner innrykk
\setlength{\parskip}{1em}   % Legger til mellomrom mellom avsnittene
\textbf{Background and aims.} In recent decades, cognitive science has made great strides in understanding the mechanisms underlying skill learning. This knowledge has inspired ideas on how to achieve better learning by leveraging these mechanisms more effectively. Thus far, most studies have focused on simple, laboratory-based tasks. This raises questions about the extent to which these learning principles can be generalized to real-world learning contexts, such as sports, education, or rehabilitation. Consequently, there is a pressing need for studies with greater ecological validity to determine whether these principles can enhance critical life functions or skills that are important to individuals. The overall goal of this doctoral thesis was to bridge this gap by using alpine ski racing as a testbed to study skill learning in skilled and elite athletes.

To achieve this goal, this doctoral thesis adopted an interdisciplinary approach grounded in both mechanics and psychology. This interdisciplinary approach was taken to learn about strategies to enhance the performance of skilled alpine skiers, and this knowledge was used as leverage to examine whether effective learning strategies could improve the learning of these strategies. Specifically, from a mechanical perspective, I asked what are the most effective strategies for skiing faster on flat slopes (research aim \RNum{1}; Paper \RNum{3}) and what the kinematic signatures of one of the strategies is that it is so effective (research aim  \RNum{2}; Paper \RNum{2}). From a psychological perspective, I asked whether skiers could learn better by using contextual interference to train learning problems (research aim 3; Paper \RNum{1}) and whether reinforcement learning is a more effective learning signal for instruction and feedback than traditional instruction-based learning (research aim 4; Paper \RNum{3}). 


%To achieve this goal, this doctoral thesis adopted a cross-disciplinary research approach involving mechanics and psychology to both learn about strategies that can improve performance in this skilled cohort of skiers and determine whether effective teaching strategies can enhance the learning and execution of these strategies. 

\textbf{Methods.} This thesis is based on two learning studies. In Study \RNum{1}, we tested whether training with a high degree of contextual interference could enhance learning in 66 skilled alpine skiers. To test this prediction, we contrasted two learning groups who learned to use the "extend" (or pumping) strategy on flat sections in slalom: an interleaved learning group who trained the strategy on three slalom courses each day in a random order (skiers never repeated more than two consecutive runs on the same course) and a blocked learning group who trained the same strategy on one slalom course per day (with a counterbalanced order of courses between skiers). The experiment spanned eight days, starting with a baseline test on the first day, followed by three days of training, and a retention test three days after the final training day. The effects of the learning groups on learning and performance are reported in Paper \RNum{1}. For three of the ski teams (18 skiers) who participated in this study, we additionally recorded the skiers' positions in the slalom course using a local positioning system during the baseline and retention tests. The purpose of this recording was to analyze changes in kinematics throughout the intervention to better understand the kinematic signature of the "extend" (pumping) strategy. The results of this analysis are reported in Paper \RNum{2}.  

In Study \RNum{2}, we tested which teaching signal for instruction and feedback is most effective for learning to choose optimal strategies. This experiment involved 98 skilled and elite alpine ski racers and compared three learning groups on their learning to make effective strategic choices to ski faster on flat sections in slalom. In the supervised (free choice) learning group, skiers were told by their coaches which strategy to use, either the one they believed to be the best or most appropriate for the skier. This learning group represents the conventional teaching strategy. We complemented this learning group with a supervised (target skill) learning group, where coaches instructed skiers to select the strategy that we defined as the theoretically best strategy based on mechanics and field observations of elite skiers. Since the skiers were instructed to use the theoretical optimal strategy, this learning group served as a benchmark for the upper limit of performance achievable through optimal strategy choices. In both of these supervised learning groups, the coaches saw the skiers’ race times but did not communicate them to the skiers. We compared these two learning groups to a reinforcement learning group where the skiers learned the values of the strategies by trying a strategy and learning from evaluations (that is, race times) instead of from a coach. The effect of the learning groups on learning to make good strategy choices for performance is reported in Paper \RNum{3}. In this study, we also reported analyses comparing movement strategies, but this analysis was overshadowed by other findings and therefore placed in a supplementary table. Therefore, in this thesis, I have reanalyzed these data and reported the findings within the main text. 

\textbf{Results}. To begin with the mechanical goals of this thesis, from Study \RNum{2} (Paper \RNum{3} and additional analysis in this thesis), we found that the skiers on average achieved the fastest race times using the “extend with rock skis forward” strategy, followed by the ”extend” (pumping), ”rock skis forward” and ”stand against” strategies in the flat section. Notably, the “extend with rock skis forward” strategy was only marginally better than the “extend” (pumping) strategy, which is simpler and nearly as effective on its own. From Study \RNum{1} (Paper \RNum{2}), we found that the skiers' speed profile changed greatly after the training intervention on the "extend" (pumping) strategy. After the intervention, the skiers reached higher speeds just after the gate passage, which continued to rise until midway between the gates before declining just before the next gate. We also observed a trend where the path length increased slightly from gate to gate, although this varied significantly from turn to turn. Switching to the psychological goals of the thesis, from Study \RNum{1} (Paper \RNum{1}), we did not find evidence that the interleaved learning group performed worse or improved retention compared to the blocked learning group, indicating no convincing evidence for the contextual interference effect. On the other hand, from Study \RNum{2} (Paper \RNum{3}), we found that learning strategies through reinforcement learning provided a more effective teaching signal than conventional instruction-based teaching. Interestingly, the reinforcement learning group also descriptively outperformed the supervised (target skill) learning group, which was intended to represent the upper limit of performance achievable through optimal strategy choices. 

\textbf{Conclusion}. This thesis shows that the strategies skiers choose can have a crucial impact on their performance in slalom. We did not find compelling evidence that training with higher contextual interference (interleaved learning) enhanced skill learning. However, we found that reinforcement learning was an effective teaching signal for training athletes to make good strategy choices. Based on these findings, coaches should develop various strategies and allow athletes to learn their values, ultimately enabling athletes to select the best option. Overall, this thesis highlights that an interdisciplinary approach can be an effective approach for studying skill learning in elite athletes, and it may represent a promising approach for future research. 