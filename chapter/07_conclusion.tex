The main conclusions drawn from the work presented in this thesis are:

\begin{itemize}
    \item We found that skiers on average achieved the fastest race times using the “extend with rock skis forward” strategy, followed by the "extend", "rock skis forward" and "stand against" strategies. Notably, the “extend with rock skis forward” strategy was only marginally better than the “extend” strategy, which is simpler and nearly as effective on its own. Therefore, we recommend that skiers choose either of these two strategies to enhance their performance on flat sections in slalom (Study \RNum{2}, unpublished results) 
    \item We found that a training intervention on the "extend" (pumping) strategy left a remarkable kinematic signature on the skiers. After the intervention, skiers exhibited a more wave-like speed profile. That is, they increased their speed after passing through a gate, which continued to rise until they were approximately midway between two gates. Then, their speed decreased until the next gate, before increasing again. Besides, we observed a trend where skiers took longer paths between gates (Study \RNum{1}, Paper \RNum{2}). 
    \item We did not find evidence that the skiers learned better by increasing the frequency at which they were exposed to new learning problems (that is, interleaved practice) (Study \RNum{1}, Paper \RNum{1}).
    \item We found that learning strategies and their values through reinforcement learning provided a more effective teaching signal than conventional instruction-based coaching (Study \RNum{2}, Paper \RNum{1})
    \item The interdisciplinary approach between psychology and mechanics proved useful in studying skill learning in skilled and elite athletes.
\end{itemize}