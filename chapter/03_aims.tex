The aim of this doctoral project was twofold. On the one hand, the project aimed to develop a better understanding of how to achieve faster race times on flats in slalom and strategies to support this goal. In addition, the project aimed to understand the kinematic signatures of the "extend" (that is, pumping) strategy—likely being an important movement—by examining the kinematic changes after a training intervention on the "extend" strategy. On the other hand, this doctoral thesis aims to test whether learning theories from cognitive science can improve skill learning in sports through better problem-solving tasks and more effective teaching signals. The two overarching goals of this doctoral thesis are mutually important and have developed in parallel. Based on these goals, I have identified four research questions that the thesis aims to address:

\begin{itemize}
    \item Which strategies do skilled skiers perform best with on flat sections? (Study 2; not published as paper) 
    \item What are the kinematic changes resulting from a prolonged training intervention on the pumping strategy (or extend strategy)? (Study 1; Paper 2)
    \item How should coaches best distribute problem-solving tasks among skilled skiers? (Study 1; Paper 1)
    \item Which teaching signal is most effective for helping skiers learn to select effective strategies for skiing fast on flat sections?  (Study 2; Paper 3)
\end{itemize}









