\section{Methological considerations}
Et stor utfordring ved doktorgraden var å finne ut hvordan vi skal måle prestasjon reliabelt i alpint over tid, som for eksempel over dager, uker eller over måneder. Gjennom doktorgradsarbeidet har jeg lært en del om hvilke utfordringer man kan møte i alpint og hvordan man kan løse disse utfordringene på best mulig måte. 

Hovedutfordringen i alpint er at ytre forhold som snøunderlag og vind varierer hele tiden, og det er nettopp på grunn av denne variasjonen at det er utfordrende å måle prestasjon på en reliabel og god måte. For å begrense denne variasjonen søkte vi derfor inn i indoor ski hall for slik at vi minimerte vind og hadde mest like forhold fra treningsøkt til treningsøkt.  I tillegg vannet vi bakken før hver gruppe med skilag vi testet med unntak av noen grupper der vi ikke fant dette som den optimale løsningen. I tillegg brukte vi straight gliding som en strategi slik at skiløperne at vi kunne kontrollere for variasjoner i snøen hos utøvere. I tillegg ba vi utøverne om å bruke samme utstyr for hver økt, som at alle alltid kjørte i fartsdress hele tiden og de prepared their skis the same way all the time. I tillegg brukte forsøkte vi å være så nøyaktige vi kunne ved å bruke målebånd. Samlet sett skal dette ha gitt en arena der vi kan måle prestasjon så godt vi kan. 

Det vi imidlertid finner når vi analyserer rett ned tidene er at, selv om vi 
