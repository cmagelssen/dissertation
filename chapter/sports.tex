Humans show a remarkable capacity to acquire and refine skills through training. For example, we can learn to ski at sufficient level after just a few hours of practice. But then, we can also continue to improve this skill to an impressive level with additional training, such as skiing carved turns on icy steep slopes. Understanding the machinery that enables this learning and identifying the mechanisms that support our ability to learn new skills and refine existing ones has long been a mystery. 

In recent years, cognitive scientists have begun to understand the mechanisms that support this learning and how various types of training can target these mechanisms specifically. What this research unfolds thus far, is that learning a skill likely engages multiple subsystems (or learning mechanisms) rather than acting on single mechanism and that the specific subsystems most engaged during learning depend on the task being learned and the particular training regimen employed. 


Som en konsekvens vet vi mye om motorisk læring. Det vi imidlertid vet mindre om er hvordan disse læringsmekanismene kan utnyttes i større grad for å lage bedre trening som har applikasjon for real world learning. Enkelte forskere har dermed forsøkt å oppmuntre til å utvikle læringsoppgaver som i større ligner vi real world tasks. Men å bruke disse oppgavene til å forstå skill learning vil på en annen side aldri kunne forklare hvordan mennesker






Recent advances in cognitive science have greatly enhanced our understanding of the mechanisms that enable this learning. Research reveals that training does not enhance a single mechanism but engages a complex system composed of multiple subsystems or learning mechanisms. The specific subsystems most engaged during learning depend on the task being learned and the specific training regimen employed. 



The contribution of each of these forms of learning to overall skill acquisition depends on the task that the learner must accomplish and the learning conditions to which they are exposed.





Humans show a remarkable capacity to acquire and refine skills through training. We can, for example, learn to drive a car but can also extend this skill to et imponerende høyt nivå, som å kjøre en race car eller formula 1 bil. In recent years, cognitive science has made great strides in understanding the mechanicms that makes this learning possible and how have various types can augment or supress these mechanism. This emerging understanding suggests that the training does not enhance a single mechanism but likely engages a complex system composed of multiple subsystems or learning mechanisms that contribute to overall learning. 



In recent decades, cognitive science has made great strides in understanding the mechanisms that govern this type of learning and how different types of training can modulate these processes for the better or worse. Slik forskere begynner å forstå systemet er at treningen vi gjør mest sannsynlig ikke bare påvirker ett system, men at motorisk læring handler om flere systemer som samhandler for å skape net learning. 

umans exhibit a remarkable capacity to acquire and refine skills through training. Recent advances in cognitive science have greatly enhanced our understanding of the mechanisms underlying this type of learning. Research indicates that training likely influences multiple systems, with motor learning involving the interaction of several systems to create effective learning outcomes. By examining how different types of training modulate these processes, scientists are beginning to unravel the complexities of motor learning. This approach not only underscores the multifaceted nature of skill acquisition but also provides insights into optimizing training methods for better performance and learning efficiency. 

different training methods guide these processes.  

 


Humans show a great capacity for learning new and refining existing skills through training. I de siste tiårene har kognitiv science gjort store fremskritt i å forstå de underliggende mekanismer that exert control of this type of learning, og hvordan ulike treningstyper guides disse læringsprosessene. These studies have revealed that motor learning is not a single system but rather comprises multiple interacting systems that support and contribute to net motor learning \cite{haith_model-based_2013, uehara_learning_2018, doya_complementary_2000, spampinato_multiple_2021,makino_circuit_2016, huang_rethinking_2011, wolpert_principles_2011, wolpert_motor_2010}. The contribution of each of these forms of learning to overall skill acquisition depends on the task that the learner must accomplish and the learning conditions to which they are exposed. Det aller meste av dette arbeidet er gjort på relativt enkle oppgaver, som . Som en konsekvens vet vi mye om motorisk læring. Det vi imidlertid vet mindre om er hvordan disse læringsmekanismene kan utnyttes i større grad for å lage bedre trening som har applikasjon for real world learning. Enkelte forskere har dermed forsøkt å oppmuntre til å utvikle læringsoppgaver som i større ligner vi real world tasks. Men å bruke disse oppgavene til å forstå skill learning vil på en annen side aldri kunne forklare hvordan mennesker  


Elite sports demand some of the most challenging learning exercises for the brain and therefore provide an exciting arena to test learning theories. In sports, skilled performers spend years training to perfect their skills to maximize performance across diverse situations and conditions. This contrasts with simpler laboratory tasks, where researchers know the best solution, and performers peak their performance after just minutes, hours, or days of training. However, studying skill learning in skilled performers presents its own methodological challenges. First, skilled performers, by definition, are already proficient and show little improvement by repeating their already automated solution with further training. That said, this lack of improvement does not necessarily mean that they have reached the real limits of their performance. Instead, it may reflect spurious limits due to a lack of knowledge about better methods to solve the task. Discovering better strategies for these performers could help them overcome these perceived limitations. For det andre er idretten en viktig og betydningsfull aktivitet for disse utøverne, så med mindre de ser at de kan bli bedre av å delta i læringseksperimentet vil de være nøkterne til å delta. Dermed er det grunnleggende viktig å faktisk utvikle områder der utøvere kan utvikle seg og utvikle strategier for å forbedre prestasjonen på disse områdene. 

Alpint er en av de mest komplekse idrettene vi har, men der vi også har objektive mål for å uttrykke prestasjon.3






mean they have reached the true limits of their abilities. 


high level of performance often shows little noticeable improvement with further training




However, 




studying skilled performers represent their own unique challenges. Because their already high performance level, their performance will unlikely improve with further training unless there is something are elements in their training that can accelerate their performance.

However, studying skilled performers poses a unique challenge because their high level of performance often shows little noticeable improvement with further training, which makes it challenging to 


. However, this high performance does not necessarily mean they have reached the true limits of their abilities. Instead, it may reflect perceived limits due to a lack of knowledge about better methods for task execution. To overcome these limits, athletes need to acquire strategies that can elevate their performance to new heights . 








Elite sports represent one of the most demanding learning challenges for our brain, and therefore serve as a severe test of these learning mechanisms. Achieving elite-level performance requires years of training to continuously refine skills and find optimal solutions across a wide range of situations and conditions. This contrasts with simpler laboratory tasks, such as button presses and reaching tasks, where peak performance is achieved after just minutes, hours, or days of training. However, studying skilled performers poses a challenge because their performance is already at a high level, and further training often does not lead to noticeable improvements. However, high performance does not necessarily mean athletes have reached the true limits of their abilities (asymptote performance). Instead, it may reflect spurious limits due to a lack of knowledge about better methods for task execution \textbackslash{}cite\{thorndike\_educational\_1913, gray\_plateaus\_2017, seagull\_expanding\_1994\}. Overcoming these limits requires knowledge of strategies that can elevate athletes to new heights.

 













Elite sports representerer en av de mest krevende læringsoppgavene vi kan eksponere hjernen vår for, og representerer en ekstrem test av disse læringsmekanismene. Grunnen til dette er at elite sportsoppnåelser krever flere år med trening for kontinerlig refinement av skills for å finne optimale løsninger i en stor range av situasjoner og conditions. Dette er i kontrast til enklere laboratorie oppgaver som knapptrykking og strekkbevegelser der asymptote prestasjon oppnås etter minutter, timer eller dager med trening. Utfordringen med å studere gode utøvere er at prestasjonen deres allerede er på et høyt nivå, og at videre trening ikke fører til økt prestasjon. Men at utøveres prestasjon er på høyt nivå ensbetyr ikke at de prestasjonen er nådd asymptote (real limits in performance), men kan gjenspeile spurious limits som kan komme av at utføreren ikke er kjent med bedre metoder for å løse oppgaven \cite{thorndike_educational_1913, gray_plateaus_2017, seagull_expanding_1994} . For å overkomme dette kreves derfor kunnskap om strategier om hvilke strategier som kan løfte utøvere til nye høyder. 


killful, adaptive strategy selection is a key characteristic of expertise \cite{ericsson_scientific_1998, ericsson_development_2003, krakauer_motor_2019, stanley_motor_2013}. Yet we know little about which teaching methods are most effective in stimulating learners to make good strategy choices, let alone the sources of information that drive these learning processes \cite{taylor_cerebellar_2014, taylor_role_2012}.








Til forskjell fra enklere oppgaver utført i labben 


Elite sports represent one of the toughest human activities and the greatest challenges for the brain \cite{walsh_is_2014}, og er derfor en ekstrem test av disse læringsmekanismene. 




Unlike simpler aboratory-based learning tasks, which typically take minutes, hours, or days to learn \cite{du_relationship_2022, yarrow_inside_2009}, becoming skilled in sports requires many years of dedicated practice \cite{hodges_predicting_2004, vaeyens_talent_2009}.  However, simply putting in the hours of practice is not enough to attain elite accomplishment. That is, skilled performers will unlikely improve much if these hours only goes into performing more repetition of their automated solution\cite{ericsson_development_2003, ericsson_role_1993, ericsson_scientific_1998, ericsson_expert_1994, du_relationship_2022}. Instead, what skilled performers must make to achieve greater leaps in performance is to break up their automated solution and compile a new and better version that consists of a more effective movement pattern, provided that such a better solution exists\cite{du_relationship_2022}. The challenge lies in identifying such effective strategies, which is difficult because we do not always know if they exist \cite{gray_plateaus_2017, cohen_effect_2021}. Therefore, studying skill learning in skilled sport performers demands a deep understanding of the sport to identify training elements that can improve performance and thus be used as training stimuli in a learning intervention. 