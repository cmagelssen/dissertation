Doktorgradsprosjektet har hatt to overordnede mål. For det første var det å utvikle en bedre forståelse for hvilke strategier som er effektive for å kjøre raskt på flater for gode utøvere. Videre var det å forstå de kinetiske endringene etter en treningsintervensjon på pumping for å bedre forstå effekten av denne strategien. For det andre var målet å undersøke om kunnskap fra kognitiv vitenskap kan forbedre ferdighetslæring gjennom bedre problemoppgaver og teachingstrategier. These two primary goals have led to four research questions that the dissertation aims to address:

\begin{itemize}
    \item What strategies do skilled skiers perform best with on flat sections? (Study 2; not published in a paper)
    \item What are the kinematic changes resulting from a prolonged training intervention on the pumping strategy (or extend strategy)? (Study 1; Paper 2)
    \item How should coaches best distribute problem-solving tasks among skilled skiers? (Study 1; Paper 1)
    \item Which teaching signal is most effective for learning strategies to ski fast on flat sections? (Study 2; Paper 3)
\end{itemize}









