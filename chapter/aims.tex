This doctoral thesis pursued two overarching aims. First, it aimed to discern what skilled skiers can do to ski faster on flat slopes in slalom. Second, it aimed to test whether coaches can help skiers make better strategy decisions and consequently also performance with reinforcement learning as opposed to traditional instruction, as well as testing whether coaches can improve skill learning with contextual interference. Together, these two goals contributes to elucidating the methodologies employed with high-level athletes. The four specific aims of the doctoral thesis are outlined below: 

\begin{enumerate}
    \item Understanding which strategies are effective for proficient skiers to achieve fast times on flat terrain in slalom.
    \item Understanding the kinematic changes following the pumping intervention (i.e., extending the strategy).
    \item Testing whether coaches should employ reinforcement learning to assist athletes in making effective choices and thereby enhancing performance.
    \item Testing whether coaches should utilize contextual interference to enhance learning in alpine skiing.
\end{enumerate}