Slalom racers' goal is to ski a course in the shortest time possible. To accomplish this goal, they need effective strategies. Et parti av en slalomløype hvor det historisk sett oppstår store tidsforskjeller er den flate seksjonen av en slalåmløype.  


Typically, they learn which strategies to select from a coach, such that a coach tells the skier to always ski the shortest line possible. Spørsmålet er imidlertidig om denne instruksjonsmetoden er den mest effektive læringsstrategien å hjelpe utøvere til å velge gode valg. Det samme spørsmålet kan man stille om trenernes løypesetting. Ofte 

Det overordnede målet med denne doktorgraden er 1. å forstå hvordan man skal kjøre raskere på flater og 2. hvordan man skal lære utøvere å kjøre raskere på flater. 








%In alpine ski racing, the goal is to ski a slalom course in the shortest time possible. Descents can be executed in nearly any way, as long as the athlete passes on the correct side of the gates placed along the slope. This aspect introduces a significant degree of freedom in technique selection. To address this freedom of choice, researchers have sought to identify the strategies and techniques employed by athletes to tackle this challenge.

%Understanding these strategies is crucial knowledge for coaches to enhance their training practices. It helps coaches identify which techniques and strategies may be beneficial to emphasize in training sessions. However, a key challenge in this research is the diverse nature of alpine skiing situations. Principles derived from one scenario may not necessarily apply to another. This highlights the need for a nuanced approach, recognizing that what works in one situation may not be universally applicable to others.



%I alpint er målet å kjøre raskest mulig ned en løype på tid. Nedkjøringer kan gjøres på nesten alle mulige måter så lenge utøveren passerer på riktig side av porten som er plassert nedover bakken, og innbyr derfor til et stort frihetsgradsproblem. For å dette frihetsgradsproblemet har forskere forsøkt å finne ut hvilke strategier og teknikker utøvere bruker for å løse dette frihetsgradsproblemet. Dette er viktig kunnskap som trenere kan bruke i sin trenerpraksis, som å vite hvilke teknikker og strategier det kan være fornuftig å stimulere til i treningsarbeidet. En utfordring med dette arbeidet er at situasjonene i alpint er svært forskjellige, slik prinsipper fra en situasjonen ikke nødvendigvs gjelder en annen. Det betyr 


%. Derfor har forskere forsøkt å studere hvilke mekanismer som er lagt til grunn for dette arbeidet. En utfordring med dette arbeidet er at situasjonene i alpint, som betyr at det vanskelig å studere noe i en situasjon. Derfor er det viktig å belyse teknikk, men det er like mye 

%Det ekspert performance approach er en teoretisk tilnærming som omhandler hvordan gir et teoretisk rammeverk for å studere alpint. Rammeverket går ut på. Med utgangspunkt i dette teoretiske rammeverket har denne studien forsøkt å finne ut hva som er en. Vi fokuserer på tekniske ferdigheter.
