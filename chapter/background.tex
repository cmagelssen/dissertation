Motor learning underpins all forms of learning \cite{shadmehr_computational_2004}. Every time we take a sip of a coffee, drive a car, or play sports, we engage our brain's evolutionary oldest learning system. This learning system was designed to help organisms navigate their environment and evolved long before vertebrates developed other capacities, such as abstract reasoning or language. Therefore, the motor learning system likely serves as a foundational building block on which later evolved learning systems are based. Studying motor learning can likely provide insight into the general learning principles that the brain uses and that underlie other forms of learning \cite{shadmehr_computational_2004}.

One impressive facets of humans is how far we can push this motor learning system to acquire an impressive set of skills. For instance, we can learn to dribble past an opponent while scanning the field to determine the best course of action. Despite our ease in recognizing skilled performances when we see it,  scientists have yet to reach a consensus on a definition of motor skill, at least one that truly improves our understanding of the subject\cite{du_relationship_2022, shadmehr_computational_2004}. However, many scientists agree that motor skill likely comprises multiple facets or components that together constitute motor skill\cite{wolpert_principles_2011, wolpert_motor_2010, wolpert_perspectives_2001, du_relationship_2022, chen_effects_2018, diedrichsen_motor_2015, stanley_motor_2013, gallivan_decision-making_2018, krakauer_motor_2019, makino_circuit_2016}. One such facet of motor skill is making good decisions \cite{gallivan_decision-making_2018, du_relationship_2022, wolpert_motor_2010}. Consider, for example, a downhill skier faced with a critical decision approaching the next gate. The dilemma involves choosing between a shorter, direct path requiring high braking forces and sacrificing speed, and a longer route promising greater velocity but covering a longer path. Through practice and experience, the skier can learn that a longer path often leads to a better outcome and choose this strategy \cite{supej_differential_2008, lesnik_best_2007, federolf_quantifying_2012}. However, choosing the best strategy among several possible options is not enough if the downhill skier cannot also execute this strategy well, such as skiing cleanly rather than skidding \cite{reid_kinematic_2010, reid_turn_2009}. Therefore, another facet is also performing the action as effectively as possible \cite{du_relationship_2022, wolpert_perspectives_2001}.

In the last decades, cognitive science have made great strides in how different types of training affect learning of these skill components (for reviews, see \cite{krakauer_motor_2019, spampinato_multiple_2021}). These studies have revealed that motor learning is not a single system but rather comprises multiple interacting systems that support and contribute to net motor learning \cite{haith_model-based_2013, uehara_learning_2018, doya_complementary_2000, spampinato_multiple_2021,makino_circuit_2016, huang_rethinking_2011, wolpert_principles_2011, wolpert_motor_2010}. These forms of learning include error-based learning, reinforcement learning, use-dependent learning, and cognitive strategies\cite{spampinato_multiple_2021}. The contribution of each of these forms of learning to overall skill acquisition depends on the task that the learner must accomplish and the learning conditions to which they are exposed. Many laboratory studies have attempted to isolate these learning mechanisms, yet it is unlikely that complete, pure isolation occurs\cite{spampinato_multiple_2021}. The big question and ultimate lie in whether any of these learning mechanisms can be leveraged more effectively to accelerate learning in the real world.

Elite sports represent one of the toughest human activities and the greatest challenges for the brain \cite{walsh_is_2014}, og er derfor en ekstrem test av disse læringsmekanismene. Unlike simpler aboratory-based learning tasks, which typically take minutes, hours, or days to learn \cite{du_relationship_2022, yarrow_inside_2009}, becoming skilled in sports requires many years of dedicated practice \cite{hodges_predicting_2004, vaeyens_talent_2009}.  However, simply putting in the hours of practice is not enough to attain elite accomplishment. That is, skilled performers will unlikely improve much if these hours only goes into performing more repetition of their automated solution\cite{ericsson_development_2003, ericsson_role_1993, ericsson_scientific_1998, ericsson_expert_1994, du_relationship_2022}. Instead, what skilled performers must make to achieve greater leaps in performance is to break up their automated solution and compile a new and better version that consists of a more effective movement pattern, provided that such a better solution exists\cite{du_relationship_2022}. The challenge lies in identifying such effective strategies, which is difficult because we do not always know if they exist \cite{gray_plateaus_2017, cohen_effect_2021}. Therefore, studying skill learning in skilled sport performers demands a deep understanding of the sport to identify training elements that can improve performance and thus be used as training stimuli in a learning intervention. 

In this doctoral thesis, I have drawn inspiration from the expert performance approach \cite{ericsson_development_2003, ericsson_prospects_2002, williams_using_2017, williams_perceptual-cognitive_2005} as a framework for identifying and studying key sporting skills that are relevant for improving performance in skilled performers and leveraging this understanding to design learning experiments aimed at enhancing performance. This approach involve three stages. The first stage involves capturing experts' performance in their natural environment and seek to understand the factors that distinguish them from less skilled performers.  In the second stage, the aim is to comprehend the mechanisms that can account for these differences through relevant measurement techniques such as motion analysis or eye movement recordings. In the final stage, the goal is to understand the types of training that can account for these underlying mechanisms, which can be achieved through examining the training history of elite performers or conducting experimental studies. Together, these three phases provide a comprehensive and systematic approach to understanding skills and skill learning in sports. By gaining insights from the first two stages, researchers acquire knowledge about the skills possessed by expert performers. This knowledge allows for the creation of effective learning studies focusing on skills that are relevant and of interest for skilled performers to enhance. Consequently, this approach helps overcome the challenge of recruiting skilled individuals for learning studies \cite{farrow_chapter_2017, buszard_quantifying_2017}, as they can directly benefit from participation by improving in areas important to them.

I have chosen alpine ski racing as the focal domain for studying skills and skill learning. My choice stems from the longstanding and close collaboration between the Norwegian School of Sport Sciences and the Norwegian Ski Federation, which has promoted research beneficial to both the sporting and academic fields. Drawing upon the expert performance approach as a gateway to understand sporting skill and learning, the doctoral thesis involved three steps. First, I identified a critical section in slalom skiing characterized by substantial time differentials among skiers. My choice was to study the flat section in slalom since big time differences can arise in this section. Second, I sought to identify and understand how skiers can ski faster on flats in slalom and elucidate the underlying mechanisms driving these strategies. This comprehension served as a foundation for designing learning situations with skilled skiers, aimed at exploring the potential for more efficacious training methodologies in imparting these strategies. Therefore, this doctoral project embodies a cross-disciplinary approach involving the fusion of mechanics, psychology, and sports. 

The overarching goal of the doctoral research was to understand how skiers improve their performance in flats in slalom and how to train skiers to select effective strategies and execute them to the best of their ability.

\section{Structure of the thesis}
Etter denne generelle introduksjonen til doktorgraden vil jeg presentere bakgrunnen for studien. Dette bakgrunnskapittelet er organisert i tre overoverordnede spørsmål (løst inspirert av det expert performance approach). I det første avsnittet vil jeg gå gjennom bakgrunnen for hvorfor flater i slalåm som ferdighet å studere videre. Deretter presenterer jeg hvilke strategier som potensielt kan være effektive for å kjøre raskt på flater. Her går jeg kortfattet gjennom det mekaniske grunnlaget for effektiv skikjøring fra et mekanisk energi perspektiv. Avslutningsvis i det kapittelet presenterer jeg to læringspardigmer som potensielt kan forbedre 

stiller vi hva en effektive læringsparagigmer for å forbedre treningen til eliteutøvere


%går vi gjennom hva som skjer skiller eliteutøvere fra mindre skilled utøvere. Det neste spørsmålet er hvilke strategier som er effektive for å kjøre raskt på flater. Her går vi gjennom mekanikken for noen strategegier som kan bidra til å kjøre raskt på flater. I siste spørsmål stiller vi hva en effektive læringsparagigmer for å forbedre treningen til eliteutøvere. Til slutt forsøker vi 


%Following this general introduction, the Background presents the context of the thesis
%regarding training prescription and adaptations to resistance training. Data from two
%training interventions are presented under Methods and Results and Discussion, referred to
%a%s Study I and Study II. Results from Study I have been published as Paper I and II, and
%results from Study II are presented in Paper III. In addition to experimental results, a meta-
%analysis examining the effects of resistance training volume on muscle mass and strength
%gains is presented under Results and Discussion. Under Methodological Considerations,
%selected topics related to the experimental data used in the thesis are discussed
