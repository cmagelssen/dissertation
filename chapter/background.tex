Motor learning underpins all forms of learning \cite{shadmehr_computational_2004}. Every time we take a sip of a coffee, drive a car, or play sports, we engage the evolutionary oldest learning systems in our brain. This motor learning system was designed to help organisms navigate their environment and respond to stimuli, and it evolved long before vertebrates developed other capacities, such as abstract reasoning or language. Therefore, the motor learning system likely serves as a foundational building block on which later evolved learning systems are based. Studying motor learning can therefore provide insight into the general learning principles that the brain uses and that underlie other forms of learning \cite{shadmehr_computational_2004}.

Despite the importance of understanding motor learning, scientists have yet to reach a consensus on a definition of motor skill, at least one that truly improves our understanding of the subject\cite{du_relationship_2022, shadmehr_computational_2004}. On the other hand, many scientists agree that motor skill likely comprises multiple facets or components that together constitute motor skill\cite{wolpert_principles_2011, wolpert_motor_2010, wolpert_perspectives_2001, du_relationship_2022, chen_effects_2018, diedrichsen_motor_2015, stanley_motor_2013, gallivan_decision-making_2018, krakauer_motor_2019, makino_circuit_2016}. One such facet of motor skill is making good decisions \cite{gallivan_decision-making_2018, du_relationship_2022, wolpert_motor_2010}. Consider, for example, a downhill skier faced with a critical decision approaching the next gate. The dilemma involves choosing between a shorter, direct path requiring high braking forces and sacrificing speed, and a longer route promising greater velocity but covering a longer path. Through practice and experience, the skier can learn that a longer path often leads to a better outcome and choose this strategy \cite{supej_differential_2008, lesnik_best_2007, federolf_quantifying_2012}. However, choosing the best strategy among several possible options is not enough if the downhill skier cannot also execute this strategy well, such as skiing cleanly rather than skidding \cite{reid_kinematic_2010, reid_turn_2009}. Therefore, another facet is also performing the action as effectively as possible \cite{du_relationship_2022, wolpert_perspectives_2001}.

In recent decades, computational approaches to motor learning have made significant strides in understanding motor learning and how different types of training affect learning outcomes. These studies have revealed that motor learning is not a singular system but rather comprises multiple interacting systems that support and contribute to net motor learning \cite{haith_model-based_2013, uehara_learning_2018}. These forms of learning include error-based learning, reinforcement learning, use-dependent learning, and cognitive strategies. The contribution of each of these forms of learning to overall skill acquisition depends on the task that the learner must accomplish and the learning conditions to which they are exposed. Many laboratory studies have attempted to isolate these learning mechanisms, yet it is unlikely that complete, pure isolation occurs. The real question and ultimate test of these theories lie in whether any of these learning mechanisms can be leveraged more effectively to accelerate learning in the real world.

Elite sports represent one of the toughest human activities and the greatest challenges for the brain \cite{walsh_is_2014}. Unlike simpler laboratory-based learning tasks, which typically take minutes, hours, or days to learn, becoming skilled in sports requires many years of dedicated practice.  However, simply putting in the hours of practice is not enough to attain elite accomplishment. That is, skilled performers will unlikely improve much if these hours only goes into performing more repetition of their automated solution. Instead, what skilled performers must make to achieve greater leaps in performance is to break up their automated solution to later compile a new and better version that consists of a more effective movement pattern, provided that such a better solution exists. The challenge lies in identifying such effective strategies, which is difficult because we do not always know if they exist \cite{gray_plateaus_2017}. Therefore, studying skill learning in skilled sport performers requires a deep understanding of the sport and staying ahead of athletes' development to identify areas for improvement. 

In this doctoral thesis, I have drawn inspiration from the expert performance approach \cite{ericsson_development_2003, ericsson_prospects_2002, williams_using_2017, williams_perceptual-cognitive_2005} as a gateway for identifying and studying key sporting skills that are relevant for improving performance in skilled individuals and leveraging this understanding to design learning experiments aimed at enhancing performance. This approach consists of three stages. The first stage involves capturing experts' performance in their natural environment and seeking to understand the factors that distinguish them from less skilled performers. Next, it entails designing representative tasks that allow highly skilled individuals to demonstrate these superior attributes. In the second stage, the aim is to comprehend the mechanisms that can account for the difference between skilled and less skilled performers through relevant measurement techniques such as motion analysis or eye movement recordings. In the final stage, the focus shifts to understanding whether there are aspects of training that can account for these underlying mechanisms, which can be achieved through examining the training history of elite performers or conducting experimental studies. Together, these three phases provide a comprehensive and systematic approach to understanding skills and skill learning in sports. By gaining insights from the first two stages, researchers acquire knowledge about the skills possessed by expert performers. This knowledge allows for the creation of effective learning studies focusing on skills that are relevant and of interest for skilled performers to enhance. Consequently, this approach helps overcome the challenge of recruiting skilled individuals for learning studies \cite{farrow_chapter_2017, buszard_quantifying_2017}, as they can directly benefit from participation by improving in areas important to them.

In my doctoral thesis, I chose alpine ski racing as the focal domain for studying skills and skill learning. My choice stems from the longstanding and fruitful collaboration between the Norwegian School of Sport Sciences and the Norwegian Ski Federation, which has facilitated investigations beneficial to both the sporting and academic fields. Drawing upon insights from the expert performance approach, the doctoral thesis involved three steps. First, I  pinpointed a critical section in slalom skiing characterized by substantial time differentials among skiers. This is the flat section in slalom. Second, I sought to identify and understand how skiers can ski faster on flats in slalom and elucidate the underlying mechanisms driving these strategies. This comprehension served as a foundation for designing learning situations with skilled skiers, aimed at exploring the potential for more efficacious training methodologies in imparting these strategies. Therefore, this doctoral project embodies a cross-disciplinary approach involving the fusion of mechanics, psychology, and sports. 
 

\section{Structure of the thesis}
Etter denne generelle introduksjonen til doktorgraden vil jeg presentere bakgrunnen for studien. Dette bakgrunnskapittelet er organisert i tre overoverordnede spørsmål (som er løst ledet ut fra det expert performance approach). I det første avsnittet går vi gjennom hva som skjer skiller eliteutøvere fra mindre skilled utøvere. Det neste spørsmålet er hvilke strategier som er effektive for å kjøre raskt på flater. Her går vi gjennom mekanikken for noen strategegier som kan bidra til å kjøre raskt på flater. I siste spørsmål stiller vi hva en effektive læringsparagigmer for å forbedre treningen til eliteutøvere. Til slutt forsøker vi 


Following this general introduction, the Background presents the context of the thesis
regarding training prescription and adaptations to resistance training. Data from two
training interventions are presented under Methods and Results and Discussion, referred to
as Study I and Study II. Results from Study I have been published as Paper I and II, and
results from Study II are presented in Paper III. In addition to experimental results, a meta-
analysis examining the effects of resistance training volume on muscle mass and strength
gains is presented under Results and Discussion. Under Methodological Considerations,
selected topics related to the experimental data used in the thesis are discussed
