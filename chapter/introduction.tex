Humans show a remarkable capacity to acquire and refine skills through training. Yet, there is a skill level that the fewest of us achieve. For example, most of us can learn to ski at an acceptable level after just a few hours of practice. but few of use can...How we are able to learn such skills and whether we can expedite this learning process from beginners to expertd through more effective training strategies has long been a topic of interest among scientists in many domains.  


For example, we can learn to ski at an acceptable level after just a few hours of practice. Then, we can also transcend this beginner level with further extensive practice, such as making carved turns on ics and steep slopes. How we are able to learn such skills and whether we can expedite this learning process through more effective training strategies has long been a topic of interest among scientists in many domains.  

Recently, cognitive scientists have made great strides in understanding the machinery that subserves this learning process and have begun to suggest ways to expedite learning by leveraging these mechanisms more effectively \cite{wolpert_principles_2011, makino_circuit_2016, spampinato_multiple_2021, krakauer_motor_2019, haith_model-based_2013}. However, most studies have focused on simpler laboratory-based tasks—button pressing or reaching movements—that enable scientists to dissect and closely examine the machinery and its various learning mechanisms. While these tightly controlled tasks offer insights into the specific contributions of individual learning mechanisms, they likely do not capture the full complexity of real-world learning\cite{krakauer_motor_2019, mangalam_investigating_2023, du_relationship_2022, chen_effects_2018, wolpert_principles_2011, gallivan_decision-making_2018}. The unanswered question, therefore, is whether the insights from these learnings have brought us closer to understanding and improving the learning of real-world skills, which engages all learning mechanisms simultaneously. Bridging the gap between laboratory and real-world learning has been identified as essential for future research to better understand whether these theories can inform training strategies for future learners  \cite{du_relationship_2022, wolpert_motor_2010, yarrow_inside_2009}. 

Elite sports offer an exciting test domain for these theories because they present one of the most demanding learning challenges for the brain \cite{walsh_is_2014}. First, elite athletes must acquire skills that optimize performance across various situations and conditions\cite{mangalam_investigating_2023, du_relationship_2022, krakauer_motor_2019}, often without a known optimal solution. The need for this type of skill learning is one reason why achieving elite status requires many years of training, in stark contrast to simpler laboratory tasks where participants reach peak performance after just minutes, hours, or days of training, and researchers already know the best solution. Second, sports involve full-body movements, unlike laboratory tasks, which typically involve only one or a few degrees of freedom. Finally, in sports, the skill is not a trivial activity learned to achieve a monetary reward; it generally holds intrinsic value for the athlete, as it constitutes an important part of their life. Therefore, athletes are generally motivated to improve their skills. Together, these elements make elite sports a unique and compelling domain for testing these learning theories.

Studying skill learning in skilled performers presents its own methodological challenges, however. Skilled performers, by definition, already possess a high level of skill and are expected to show minimal improvement by repeating their already automated solutions with further training. This makes it difficult to achieve the skill progression necessary to study motor learning. This should not be mistaken to say that it is impossible or that skilled performers cannot improve. Oftentimes, their lack of improvement simply reflects a spurious limit due to a lack of knowledge about better methods to solve the task. The discovery of better strategies for these athletes could help them overcome these limitations. Additionally, sports hold significant value for these athletes, so unless they see the learning experiment as beneficial for their improvement, they may be reluctant to participate. Therefore, it is essential to find areas where athletes can improve and create strategies to enhance performance in these areas. By doing so, researchers can ensure that athletes are motivated to engage in learning experiments and will sacrificy their normal training to participate.

Some sports have better characteristics suited as test domains due appropriate ways of quantififying performance, while also incorporating elements that cannot be studied in a laboratory setting. In these respects, alpine skiing is an ideal sport because performance is measured by the time a skier takes to complete a course, which gives an objective way to quantify performance and learning. Additionally, it demands continuous adaptation of strategies to achieve goals due to the constantly changing course settings, terrain, equipment, snow conditions, and weather. These variables make it nearly impossible for coaches to determine the best strategy at any given time, making alpine skiing an excellent subject for study. Because of these characteristics, I have chosen alpine ski racing as the focal domain for studying skills and skill acquisition for my doctoal project. A secondary reason is the longstanding and close collaboration between the Norwegian School of Sport Sciences and the Norwegian Ski Federation, which has fostered mutual knowledge development about the sport.

To address these challenges, this dissertation uses the expert performance approach as a backbone to systematically study skills and learning in this skilled cohort of performers. The first step in this approach is to identify areas or situations that distinguish between high and low performers, or where significant differences arise. This step is followed by understanding the mechanisms behind these differences and, finally, exploring the learning processes that explain them. In this dissertation, I have selected slalom sections as a central skill because quantitative data from elite performers shows significant variation in performance within these sections.

The next phase of the expert performance approach involves understanding the mechanisms that explain why these differences occur. Although we currently have limited knowledge about this, I have reviewed the mechanics literature and found quantitative evidence. Finally, I have applied this knowledge to improve performance by identifying and implementing effective learning processes.

This structured approach not only highlights the key factors distinguishing top performers but also delves into the underlying reasons for these differences. By systematically studying these elements, we can develop targeted training methods that enhance skill acquisition and performance in slalom skiing.

 



For å takle disse utfordringene har jeg i denne doktorgraden brukt det ekspert performance approach as a backbone for å systematisk studere ferdigheter og læring i denne skilled kohort of performers. Det første leddet i denne tilnærmingen går ut på først å identifisere områder eller situasjoner som skiller gode og mindre gode utøvere eller situasjoner der det oppstår store forskjeller for deretter å forstå mekanismene for hvorfor disse oppstår og til slutt hvilke læringsprosesser som kan forklare disse. I doktorgraden har jeg valgt ut flater i slalåm som en sentralt ferdighet fordi det er kvantitaiv data fra eliteutøvere som viser at denne seksjonen skiller mye. Den neste fasen i det ekspert performance approach er å forstå mekanismene som kan forklare hvorfor disse forskjellene oppstår, som i mitt tilfelle handler om hvorfor disse forskjenne oppstår. I dag vet vi imidlertid lite om akkurat dette, men jeg har lett i mekanikklitteraturen og funnet kvantitative evidense. Til slutt har jeg brukt denne kunnskapen til å   


Deretter har jeg gått inn i mekanikken for å innhente kunnskap om hvordan man kan forbedre prestasjonene på denne seksjonen. Deretter har jeg stilt spørsmålet hvordan man ka. 




%kombinert mekanikk og psykologi for å finne ferdighetsområder der utøvere kan forbedre sine ferdigheter og 


 
 












% som regel utøvere motivert for å lære fordi ferdighet


%demand some of the most challenging learning exercises for the brain and therefore provide an exciting arena to test learning theories. In sports, skilled performers spend years training to perfect their skills to maximize performance across diverse situations and conditions. This contrasts with simpler laboratory tasks, where researchers know the best solution, and performers peak their performance after just minutes, hours, or days of training. However, studying skill learning in skilled performers presents its own methodological challenges. First, skilled performers, by definition, are already proficient and show little improvement by repeating their already automated solution with further training. That said, this lack of improvement does not necessarily mean that they have reached the real limits of their performance. Instead, it may reflect spurious limits due to a lack of knowledge about better methods to solve the task. Discovering better strategies for these performers could help them overcome these perceived limitations. For det andre er idretten en viktig og betydningsfull aktivitet for disse utøverne, så med mindre de ser at de kan bli bedre av å delta i læringseksperimentet vil de være nøkterne til å delta. Dermed er det grunnleggende viktig å faktisk utvikle områder der utøvere kan utvikle seg og utvikle strategier for å forbedre prestasjonen på disse områdene. 

%For å 
 






%One impressive facets of humans is how far we can push this motor learning system to acquire an impressive set of skills. For instance, we can learn to dribble past an opponent while scanning the field to determine the best course of action. Despite our ease in recognizing skilled performances when we see it,  scientists have yet to reach a consensus on a definition of motor skill, at least one that truly improves our understanding of the subject\cite{du_relationship_2022, shadmehr_computational_2004}. However, many scientists agree that motor skill likely comprises multiple facets or components that together constitute motor skill\cite{wolpert_principles_2011, wolpert_motor_2010, wolpert_perspectives_2001, du_relationship_2022, chen_effects_2018, diedrichsen_motor_2015, stanley_motor_2013, gallivan_decision-making_2018, krakauer_motor_2019, makino_circuit_2016}. One such facet of motor skill is making good decisions \cite{gallivan_decision-making_2018, du_relationship_2022, wolpert_motor_2010}. Consider, for example, a downhill skier faced with a critical decision approaching the next gate. The dilemma involves choosing between a shorter, direct path requiring high braking forces and sacrificing speed, and a longer route promising greater velocity but covering a longer path. Through practice and experience, the skier can learn that a longer path often leads to a better outcome and choose this strategy \cite{supej_differential_2008, lesnik_best_2007, federolf_quantifying_2012}. However, choosing the best strategy among several possible options is not enough if the downhill skier cannot also execute this strategy well, such as skiing cleanly rather than skidding \cite{reid_kinematic_2010, reid_turn_2009}. Therefore, another facet is also performing the action as effectively as possible \cite{du_relationship_2022, wolpert_perspectives_2001}.



%Elite sports represent one of the toughest human activities and the greatest challenges for the brain \cite{walsh_is_2014}, og er derfor en ekstrem test av disse læringsmekanismene. Unlike simpler aboratory-based learning tasks, which typically take minutes, hours, or days to learn \cite{du_relationship_2022, yarrow_inside_2009}, becoming skilled in sports requires many years of dedicated practice \cite{hodges_predicting_2004, vaeyens_talent_2009}.  However, simply putting in the hours of practice is not enough to attain elite accomplishment. That is, skilled performers will unlikely improve much if these hours only goes into performing more repetition of their automated solution\cite{ericsson_development_2003, ericsson_role_1993, ericsson_scientific_1998, ericsson_expert_1994, du_relationship_2022}. Instead, what skilled performers must make to achieve greater leaps in performance is to break up their automated solution and compile a new and better version that consists of a more effective movement pattern, provided that such a better solution exists\cite{du_relationship_2022}. The challenge lies in identifying such effective strategies, which is difficult because we do not always know if they exist \cite{gray_plateaus_2017, cohen_effect_2021}. Therefore, studying skill learning in skilled sport performers demands a deep understanding of the sport to identify training elements that can improve performance and thus be used as training stimuli in a learning intervention. 

%In this doctoral thesis, I have drawn inspiration from the expert performance approach \cite{ericsson_development_2003, ericsson_prospects_2002, williams_using_2017, williams_perceptual-cognitive_2005} as a lens for identifying and studying key sporting skills that are relevant for improving performance in skilled performers and leveraging this understanding to design learning experiments aimed at enhancing performance. This approach involve three stages. The first stage involves capturing experts' performance in their natural environment and seek to understand the factors that distinguish them from less skilled performers.  In the second stage, the aim is to comprehend the mechanisms that can account for these differences through relevant measurement techniques such as motion analysis or eye movement recordings. In the final stage, the goal is to understand the types of training that can account for these underlying mechanisms, which can be achieved through examining the training history of elite performers or conducting experimental studies. Together, these three phases provide a comprehensive and systematic approach to understanding skills and skill learning in sports. By gaining insights from the first two stages, researchers acquire knowledge about the skills possessed by expert performers. This knowledge allows for the creation of effective learning studies focusing on skills that are relevant and of interest for skilled performers to enhance. Consequently, this approach helps overcome the challenge of recruiting skilled individuals for learning studies \cite{farrow_chapter_2017, buszard_quantifying_2017}, as they can directly benefit from participation by improving in areas important to them.

%I have chosen alpine ski racing as the focal domain for studying skills and skill learning. My choice stems from the longstanding and close collaboration between the Norwegian School of Sport Sciences and the Norwegian Ski Federation, which has promoted research beneficial to both the sporting and academic fields. Drawing upon the expert performance approach as a gateway to understand sporting skill and learning, the doctoral thesis involved three steps. First, I identified a critical section in slalom skiing characterized by substantial time differentials among skiers. My choice was to study the flat section in slalom since big time differences can arise in this section. Second, I sought to identify and understand how skiers can ski faster on flats in slalom and elucidate the underlying mechanisms driving these strategies. This comprehension served as a foundation for designing learning situations with skilled skiers, aimed at exploring the potential for more efficacious training methodologies in imparting these strategies. Therefore, this doctoral project embodies a cross-disciplinary approach involving the fusion of mechanics, psychology, and sports. 

The overarching goal of the doctoral research was to understand how skiers improve their performance in flats in slalom and how to train skiers to select effective strategies and execute them to the best of their ability.

\section{Structure of the thesis}
Etter denne generelle introduksjonen vil jeg presentere bakgrunnen for studien. I dette bakgrunnkapittelet vil jeg først redegjøre for ferdigheten jeg har valgt å studere i doktorgraden gjennom å presentere forskning som viser hvorfor denne ferdigheten er viktig. Deretter vil jeg presentere fire strategier som potensielt kan forbedre gode skikjøres prestasjon på flater gjennom mekanikk. Avslutningsvis i dette kapittelet vil jeg presentere læringsteorier som potensielt kan forbedre utøveres læring enten ved å lage problemer som utøvere skal løse eller ved å 


Etter denne generelle introduksjonen til doktorgraden vil jeg presentere bakgrunnen for studien. Dette bakgrunnskapittelet er organisert i tre overoverordnede spørsmål (løst inspirert av det expert performance approach). I det første avsnittet vil jeg gå gjennom bakgrunnen for hvorfor flater i slalåm som ferdighet å studere videre. Deretter presenterer jeg hvilke strategier som potensielt kan være effektive for å kjøre raskt på flater. Her går jeg kortfattet gjennom det mekaniske grunnlaget for effektiv skikjøring fra et mekanisk energi perspektiv. Avslutningsvis i det kapittelet presenterer jeg to læringspardigmer som potensielt kan forbedre 

stiller vi hva en effektive læringsparagigmer for å forbedre treningen til eliteutøvere


%går vi gjennom hva som skjer skiller eliteutøvere fra mindre skilled utøvere. Det neste spørsmålet er hvilke strategier som er effektive for å kjøre raskt på flater. Her går vi gjennom mekanikken for noen strategegier som kan bidra til å kjøre raskt på flater. I siste spørsmål stiller vi hva en effektive læringsparagigmer for å forbedre treningen til eliteutøvere. Til slutt forsøker vi 


%Following this general introduction, the Background presents the context of the thesis
%regarding training prescription and adaptations to resistance training. Data from two
%training interventions are presented under Methods and Results and Discussion, referred to
%a%s Study I and Study II. Results from Study I have been published as Paper I and II, and
%results from Study II are presented in Paper III. In addition to experimental results, a meta-
%analysis examining the effects of resistance training volume on muscle mass and strength
%gains is presented under Results and Discussion. Under Methodological Considerations,
%selected topics related to the experimental data used in the thesis are discussed
