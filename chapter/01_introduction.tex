
Humans show a remarkable ability to learn new skills, such as driving a car or skiing down an easy slope after just a few hours of practice. Beyond these everyday skills, there also exists an upper echelon of performance that only the fewest of us attain—such as playing in the Wimbledon tennis finals—and that is achievable only through many years of dedicated practice \parencite{hodges_predicting_2004, ericsson_role_1993, vaeyens_talent_2009, ericsson_expert_1994, ericsson_scientific_1998}. Skill learning thus spans a continuum from initial competence to elite performance, with the latter demanding extensive hours of dedicated practice. But to make the leap from these initial skill levels to elite performance, it is simply not enough to practice. Somewhere during the transition toward elite performance, many learners stop improving and consequently fail to unleash their true potential because they hit plateaus and lack knowledge of solutions to better solve the task \parencite{thorndike_educational_1913, grayloooooong,grayshort,ericsson_scientific_1998, ericsson_development_2003}. This situation raises the question of whether better training methods exist that could expedite skill learning and help learners overcome these plateaus. If such methods are available, they could significantly enhance the training of both current and future learners. 

In recent decades, cognitive science has made great strides in understanding the mechanisms that likely exert control of skill learning and how they can be leveraged to improve learning and performance \parencite{wolpert_principles_2011, makino_circuit_2016, spampinato_multiple_2021, krakauer_motor_2019, haith_model-based_2013, huang_rethinking_2011, shmuelof_are_2011, doya_complementary_2000, yarrow_inside_2009}. Most of these studies have leaned toward simple, laboratory-based learning tasks \parencite{krakauer_motor_2019, du_relationship_2022}, such as button/ball pressing \parencite{hardwick_time-dependent_2019, vassiliadis_reward_2021} or reaching movements \parencite{shadmehr_adaptive_1994, krakauer_learning_2000}, which enable scientists to dissect and closely examine the contributions of individual learning mechanisms in a controlled environment \parencite{spampinato_multiple_2021}. But their advantage also comes with a cost; they are unlikely to capture the full complexity of real-world skill learning \parencite{krakauer_motor_2019, mangalam_investigating_2023, du_relationship_2022, chen_effects_2018, wolpert_principles_2011, gallivan_decision-making_2018, iyer_probing_2020, ingram_naturalistic_2011}. Therefore, an outstanding question is whether insights from these studies have brought us closer to understanding and improving the learning of real-world skills, which engage all learning mechanisms simultaneously \parencite{spampinato_multiple_2021}. Bridging the gap between laboratory and real-world learning has thus been identified as a critical direction for future research to learn whether these theories have practical applications in natural environments, such as sports, education, and rehabilitation \parencite{du_relationship_2022, wolpert_motor_2010, yarrow_inside_2009, haar_motor_2020, ingram_naturalistic_2011, mangalam_investigating_2023, tsay_bridging_2024, spampinato_multiple_2021}.

Elite sports offer an exciting testbed for these theories because they present one of the most demanding learning challenges for the brain \parencite{walsh_is_2014}. First, elite athletes must learn to optimize their performance across many situations and conditions\parencite{mangalam_investigating_2023, du_relationship_2022, krakauer_motor_2019}. The need for this adaptability is one reason why achieving elite status requires many years of training \parencite{krakauer_motor_2019}, in stark contrast to simpler laboratory tasks where participants reach peak performance after just minutes, hours, or days of training, and researchers already know the best solution. Second, sports involve full-body movements, unlike laboratory tasks, which typically involve only one or a few degrees of freedom \parencite{du_relationship_2022}. Finally, sports are not trivial activities but generally hold intrinsic value for athletes, as they constitute an important part of their life. Therefore, athletes are generally motivated to improve their skills. Together, these characteristics make elite sports a unique and compelling testbed for testing these learning theories.

Studying skill learning in elite athletes presents its own methodological challenges, however. By definition, elite athletes are already highly skilled and are expected to show minimal improvement by repeating their automated solution with additional training\parencite{ericsson_development_2003, ericsson_expert_1994, ericsson_scientific_1998}. Consequently, learning experiments targeting familiar solutions are unlikely to bring forth improvement in performance \parencite{thorndike_educational_1913, ericsson_development_2003, grayloooooong, grayshort, ericsson_scientific_1998}, making elite sports an unsuitable domain for studying skill learning processes. However, this obstacle can be overcome by rethinking the approach taken to the learning experiments. Often, the reason elite athletes do not progress is their lack of knowledge about better methods to solve tasks \parencite{grayloooooong, grayshort, thorndike_educational_1913}. Therefore, developing strategies that help athletes improve task performance could significantly enhance their skills and is essential for testing learning theories on this skilled cohort of athletes. 

This doctoral research has chosen alpine ski racing as platform to study skill learning in skilled athletes because it servers as an excellent testbed to study skill learning due to the sport's extreme demands for skillful execution of well-chosen strategies. Alpine skiing stands out in this respect because athletes cannot perfect their performance to one situation—as is the case in many sports—but continually adjust their solutions to varying conditions such as equipment, course settings, terrain variations, snow conditions, and weather \parencite{gilgien_training_2018, supej_recent_2019, reid_kinematic_2010}. This variability makes it challenging for skiers and coaches to determine the optimal strategy at any given moment, introducing uncertainty about the strategies is a second dimension why this sport is an attractive domain to test these theories. In addition, the long-standing, collaborative relationship between the Norwegian School of Sport Sciences and the Norwegian Ski Federation has enriched mutual understanding and knowledge about the sport, further justifying its selection for this doctoral project. 

To use alpine skiing as a bridge between basic skill learning and real-world skill learning, this doctoral project has adopted a cross-disciplinary approach based on mechanics and psychology. The two disciplines are fusioned in the following way: First, I identified a critical section of a slalom course where significant time differences occur between elite skiers and delineated strategies that could improve the performance of skilled skiers based on mechanics and quantitative observations of elite skiers in the field. Then, I set out to test these strategies and to understand the signatures they exert on skiers' kinematics as a way to further improve our understanding of these strategies. Second, I used this knowledge to build interventions to improve skiers' performance in this section by leveraging these strategies to test learning theories from cognitive science with skilled skiers. As such, the doctoral thesis had a dual research strategy goal where the goal, on the one hand, was to better understand what strategies skiers can use to make them faster in this critical section and why these strategies are effective, while, on the other hand, test if we can use principles from learning theories in cognitive science to better teach and train these strategies. In this doctoral thesis, the two objectives have been studied in parallel and have mutually informed each other both across and within individual studies. 

\section{Structure of the thesis}
The structure of this doctoral thesis is closely organised around these two main goals. Following this general introduction, I will present the background for the doctoral project. In this background chapter I will first argue for the specific course section we have focused on in our doctoral research. Next, I present strategies that could enhance the performance of skilled skiers in this section. In the final section of that chapter, I will introduce learning theories from cognitive science that have proven effective in simple laboratory tasks, and that may hold promise for coaches in this sport. This introduction and background chapter concludes with the research aims and questions that I seek to address in this doctoral thesis.The doctoral thesis comprises two learning experiments, referred to as Study \RNum{1} and Study \RNum{2}. Study \RNum{1} resulted in two papers, designated Paper \RNum{1} and Paper \RNum{2}. Study \RNum{2} led to Paper \RNum{3}. Additionally, data from this learning experiment were used to estimate the effects of the various strategies tested, although this analysis is not published in a paper but is presented in the results and discussion chapter of this dissertation. The results and discussion are organized around the two primary questions of the doctoral project: how can athletes improve their performance, and how should they learn these improvements? Following this discussion, I will provide a general discussion, synthesizing the studies as a whole and including a methodological discussion.
