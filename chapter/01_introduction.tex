\newcommand{\RNum}[1]{\uppercase\expandafter{\romannumeral #1\relax}}


Humans show a remarkable ability to learn skills, such as driving a car or skiing down an easy slope after just a few hours of practice. Beyond these everyday skills, there also exists an upper stage of performance that only the fewest of us attain—such as playing in the Wimbledon tennis finals—and that is achievable only through many years of dedicated practice \cite{hodges_predicting_2004, ericsson_role_1993, vaeyens_talent_2009}. Skill learning can thus be seen not only as a process of becoming competent but also as one that can be extended to achieve extraordinary skills \cite{ericsson_development_2003, ericsson_scientific_1998}. 
Unfortunately, many learners who aim for the highest stage of performance stop improving and never reach their peak due to plateaus and a lack of better solutions to solve the task \cite{thorndike_educational_1913, grayloooooong,grayshort,ericsson_scientific_1998, ericsson_development_2003}. Are there better training methods that can expedite skill learning and help learners move beyond these plateaus? If so, these training methods could greatly impact the training of present and future learners. 

In recent decades, cognitive science has made great strides in understanding the mechanisms that likely exert control of skill learning and how they can be leveraged to improve learning and performance \cite{wolpert_principles_2011, makino_circuit_2016, spampinato_multiple_2021, krakauer_motor_2019, haith_model-based_2013, huang_rethinking_2011, shmuelof_are_2011, doya_complementary_2000}. Most of these studies have leaned toward simple, laboratory-based learning tasks\cite{krakauer_motor_2019, du_relationship_2022}, such as button/ball pressing \cite{hardwick_time-dependent_2019, vassiliadis_reward_2021} or reaching movements\cite{shadmehr_adaptive_1994, krakauer_learning_2000},  which enable scientists to dissect and closely examine the contributions of individual learning mechanisms in a controlled environment \cite{spampinato_multiple_2021}. But their advantage also comes with a cost; they are unlikely to capture the full complexity of real-world skill learning \cite{krakauer_motor_2019, mangalam_investigating_2023, du_relationship_2022, chen_effects_2018, wolpert_principles_2011, gallivan_decision-making_2018, iyer_probing_2020, ingram_naturalistic_2011}. Therefore, an outstanding question is whether insights from these studies have brought us closer to understanding and improving the learning of real-world skills, which engage all learning mechanisms simultaneously \cite{spampinato_multiple_2021}. Bridging the gap between laboratory and real-world learning has thus been identified as a critical direction for future research to learn whether these theories have practical applications in natural environments, such as sports, education, and rehabilitation \cite{du_relationship_2022, wolpert_motor_2010, yarrow_inside_2009, haar_motor_2020, ingram_naturalistic_2011}.

Elite sports offer an exciting test domain for these theories because they present one of the most demanding learning challenges for the brain \cite{walsh_is_2014}. First, elite athletes must learn to optimize their performance across many situations and conditions\cite{mangalam_investigating_2023, du_relationship_2022, krakauer_motor_2019}. The need for this adaptability is one reason why achieving elite status requires many years of training \cite{krakauer_motor_2019}, in stark contrast to simpler laboratory tasks where participants reach peak performance after just minutes, hours, or days of training, and researchers already know the best solution. Second, sports involve full-body movements, unlike laboratory tasks, which typically involve only one or a few degrees of freedom \cite{du_relationship_2022}. Finally, sports are not trivial activities but generally hold intrinsic value for athletes, as they constitute an important part of their life. Therefore, athletes are generally motivated to improve their skills. Together, these characteristics make elite sports a unique and compelling domain for testing these learning theories.

Studying skill learning in elite athletes presents its own methodological challenges, however. By definition, elite athletes are already highly skilled and are expected to show minimal improvement by repeating their automated solution with additional training\cite{ericsson_development_2003, ericsson_expert_1994, ericsson_scientific_1998}. Consequently, learning experiments targeting familiar solutions are unlikely to bring improvement in performance, making elite sports an unsuitable domain for studying skill learning processes  \cite{thorndike_educational_1913, ericsson_development_2003, grayloooooong, grayshort, ericsson_scientific_1998}. However, this obstacle can be overcomed by rethinking the approach taking to the learning experiments. Often, the reason elite athletes do not progress is their lack of knowledge about better methods to solve tasks \cite{grayloooooong, grayshort, thorndike_educational_1913}. Therefore, developing strategies that help athletes improve task performance could significantly enhance their skills and is essential for testing learning theories on this skilled cohort of athletes. 
%However, this lack of improvement does not mean skilled performers cannot improve. Often, their limited performance improvement reflects a plateau caused by a lack of knowledge about better methods to solve the task. Helping athletes discover better strategies can enable them to overcome these limits and ensure performance advancement. Therefore, a necessary condition for conducting learning experiments with skilled athletes is finding ways to make them improve their skill.

This doctoral project has chosen alpine skiing as its focus due to the sport's extreme demands for skillful execution of well-chosen strategies. Alpine skiing stands out in this respect because athletes must not only adapt to one unique situation—as is the case in many sports—but continually adjust to varying conditions such as equipment, course settings, terrain variations, snow conditions, and weather. This variability makes it challenging for skiers and coaches to determine the optimal strategy at any given moment, introducing uncertainty about the strategies as a second dimension. In addition, the long-standing, collaborative relationship between the Norwegian School of Sport Sciences and the Norwegian Ski Federation has enriched mutual understanding and knowledge about the sport, further justifying its selection for this study. 

To use alpine skiing as a bridge between basic skill learning and real-world skill learning, this doctoral project adopted a cross-disciplinary approach based on mechanics and psychology. On the one hand, we identified a critical section of a slalom course where significant time differences occur between elite skiers and used mechanics to delineate strategies that could improve the performance of skilled skiers based on mechanics and quantitative observations of elite skiers in the field. We then set out to test these strategies and to understand their kinematic signatures on skiers' skiing. On the other hand, we used this knowledge to build interventions to improve skiers' performance in this section by leveraging these strategies to test learning theories from cognitive science with skilled skiers. As such, the doctoral thesis had a dual research focus where the goal was to better understand what strategies skiers can use to make them faster in this critical section and why these strategies work, as well as to understand if we can use principles from learning theories in cognitive science to better teach and train these strategies.



%Due to the necessity of finding ways to improve elite athletes' skills, this doctoral project has two overarching goals. On the one hand, it seeks to identify areas where elite athletes can improve and build knowledge and understanding of strategies they can use to achieve this. On the other hand, it aims to use this knowledge to conduct learning experiments that test cognitive science theories to enhance elite athletes' training. To synthesize these typically distinct domains of research, the project draws inspiration from the expert performance approach as a systematic framework to guide the questions for this doctoral project. This approach involves three stages. In the first stage, the focus is on identifying the areas that distinguish highly skilled from less skilled athletes. The second stage seeks to understand the mechanisms that explain these differences. Finally, the third stage examines whether these mechanisms can be explained through differences in training, which can be investigated through training history or experimental learning studies. Our approach is broader than the expert performanc approach because we aim not only to understand the mechanisms underlying elite performers' superior achievements but also to identify what athletes can do to enhance their performance and how we can improve training to achieve this. 

%The aim of the doctoral project was to identify key sections in a slalom course that differentiate skilled from less skilled skiers and then use this section to answer two overarching questions: What can skiers do to ski faster in this section, and how should we teach to best facilitate learning these skills? To test whether cognitive science learning theories can enhance training compared to traditional teaching strategies, the doctoral project focuses on current coaching practices and the opportunities coaches have to create constructive learning situations. 

\section{Structure of the thesis}
The structure of this doctoral thesis is closely organised around these two main goals. Following this general introduction, I will present the background for the doctoral project. In this background chapter I will first argue for the specific course section we have focused on in our doctoral research. Next, I present strategies that could enhance the performance of skilled skiers in this section. In the final section of that chapter, I will introduce learning theories from cognitive science that have proven effective in simple laboratory tasks, and that may hold promise for coaches in this sport. This introduction and background chapter concludes with the research aims and questions that I seek to address in this doctoral thesis. 

The doctoral thesis comprises two learning experiments, referred to as Study \Rnum{1} and Study \Rnum{2}. Study \RNum{1} resulted in two papers, designated Paper \Rnum{1} and Paper \Rnum{2}. Study \Rnum{2} led to Paper \Rnum{3}. Additionally, data from this learning experiment were used to estimate the effects of the various strategies tested, although this analysis is not published in a paper but is presented in the results and discussion chapter of this dissertation. The results and discussion are organized around the two primary questions of the doctoral project: how can athletes improve their performance, and how should they learn these improvements? Following this discussion, I will provide a general discussion, synthesizing the studies as a whole and including a methodological discussion.

%går vi gjennom hva som skjer skiller eliteutøvere fra mindre skilled utøvere. Det neste spørsmålet er hvilke strategier som er effektive for å kjøre raskt på flater. Her går vi gjennom mekanikken for noen strategegier som kan bidra til å kjøre raskt på flater. I siste spørsmål stiller vi hva en effektive læringsparagigmer for å forbedre treningen til eliteutøvere. Til slutt forsøker vi 


%Following this general introduction, the Background presents the context of the thesis
%regarding training prescription and adaptations to resistance training. Data from two
%training interventions are presented under Methods and Results and Discussion, referred to
%a%s Study I and Study II. Results from Study I have been published as Paper I and II, and
%results from Study II are presented in Paper III. In addition to experimental results, a meta-
%analysis examining the effects of resistance training volume on muscle mass and strength
%gains is presented under Results and Discussion. Under Methodological Considerations,
%selected topics related to the experimental data used in the thesis are discussed
