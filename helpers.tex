
Doktorgradsprosjektet er et samarbeidsprosjekt mellom the Norwegian School of Sport Sciences (department for physical performance), the Norwegian Ski Federation, SNØ (the indoor ski hall in Oslo) og alle skigymnas og klubber i Norge og Sverige. Jeg er takknemlig for muligheten dere har gitt meg til å gjøre et slikt prosjekt som dette, og jeg håper jeg har klart å levere tilbake til dere med praktisk anvennelig kunnskap, men som også er av teoretisk interesse for psykologifaget. 

Det er en rekke mennesker som skal ha en stor takk for at hjulpet.

Romy Frömer: Jeg er veldig glad for at jeg spurte deg om hvordan du hadde laget dine figurer til elife artikkel, og at du spurte meg "er det noe mer jeg kan hjelpe deg med?" etter at du hadde forklart meg hvordan. Da fikk jeg muligheten til å presentere noen tanker rundt reinforcement learning. Siden dette har vi jobbet sammen, og du ble formelt min veileder høsten. Jeg umåtelig takknemlig for måten du har veiledet meg på gjennom ukentlig mandagsmøter og tett oppfølging. Du har trent meg i vitenskapelig tenkning, open science og skriving.  

Per Haugen: Takk for all hjelp du har gjort med doktorgraden. Du har vært en god sparringspartner gjennom alle mine år på NIH 

Matthias Gilgien: Takk for at du formelt var min veileder. Takk for ditt bidrag under datainnsamling. Uten dette hadde jeg ikke kommet i mål.


Rune Velta, Jasper Bates, Andreas Lodden, Kasper Sjøstrand, Kjersti Mejdell Styrmoe, Magne, Einar Witeeween, Allan Henriksen, Tom, Victoria, Ørjan Lydersen, Eirik Knutsen, Live Luteberget, Sindre (NIH), Tobias Torjussen, Andrea IRS, Katrine, Kodern, Mai Sissel Linløkkel, Han i 3etg, Simen Tajet,William, Kjæresten til Mai Sissel, Jeremy, Tim Gfeller, Synne, Andreas Kollebborg, han andre EC, Tommy Akselsen, Tobias, Solstad, Knut Johan Bera

Jeg har også lyst til å takke gode kollegaer på Oslo Nye Høgskole. 
