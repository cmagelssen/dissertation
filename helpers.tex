
Doktorgradsprosjektet er et samarbeidsprosjekt mellom the Norwegian School of Sport Sciences (Department of Physical Performance), the Norwegian Ski Federation, SNØ (the indoor ski hall in Oslo) og skigymnas og klubber i Norge og Sverige. Jeg er takknemlig for muligheten dere har gitt meg til å gjøre et slikt prosjekt som dette, og jeg håper jeg har klart å levere tilbake til dere med praktisk anvennelig kunnskap, men som også er av teoretisk interesse for psykologifaget. 

Det er en rekke mennesker som skal har hjulpet med på veien og som fortjener en stor takk.

Romy Frömer: Jeg er veldig glad for at jeg spurte deg om hvordan du hadde laget dine figurer til din elife artikkel, og at du spurte meg "er det noe mer jeg kan hjelpe deg med?" etter at du hadde forklart meg hvordan du laget slike figurer. Det gjorde at jeg muligheten til å presentere noen tanker rundt reinforcement learning i alpint. Siden dette har vi jobbet sammen, og du ble formelt min veileder høsten 2023. Jeg umåtelig takknemlig for måten du har veiledet meg på gjennom ukentlig mandagsmøter og tett oppfølging. Du har trent meg i vitenskapelig tenkning, open science og skriving.  

Per Haugen: Takk for all hjelp du har gjort med doktorgraden. Du har vært en enestående sparringspartner gjennom alle mine år på NIH-fra bacelor til doktorgradsstudiet. Din dør har alltid står åpen, og har vært en person jeg alltid kan ringe om jeg står fast, enten det er med å forskning eller at jeg sliter med en polish som ikke vil heve. Du er den mest kunnskapsrike personen om alpint. 

Robert Reid: Takk for sparringen vi har hatt rundt prosjektene-fra bachelor til doktorgrad, og at du har støttet og forankret prosjektet i Norges skifobund sin FOU strategy. 

Thomas Losnegard: Takk for at du har vært genuint interessert i motorisk læring og har støttet prosjektet gjennom hele min doktorgrad. Dine råd om å holde ting enkelt og gjennomførbart har vært avgjørende for at jeg har kommet i mål. 

Matthias Gilgien: Takk for at du formelt var min veileder. Takk for ditt bidrag under datainnsamling. 


I tillegg er det mange hjelpere som har jobbet døgnet rundt for at jeg skal få gjort prosjekene. Flere har hatt alarmen på kl 2am  for i flere uker i strekk for å få komme tidlig nok i skihallen for å rigge klart til utøverne. 

Rune Velta, Petter Husevåg Jølstad, Jasper Bates, Andreas Lodden, Kasper Sjøstrand, Kjersti Mejdell Styrmoe, Andreas Lodden, Johan Ansnes, Magne Lund-Hansen, Einar Witteveen, Allan Henriksen, Tom Knudsen, Victoria S Placth, Sara Stensø, Ørjan Lydersen, Eirik Knutsen, Haakon Staff, Live Luteberget, Sindre Lindstad Hoholm, Aurora Haug, Tor-Magnus Blakstad Cappelen, Katrine Groth Eggar, Mai Sissel Linløkkel, Simen Leithe Tajet, Erland Vedeler Stubbe, Erland Hoff Thommasen, Jeremy Phyffer, Tim Gfeller, Andreas Kollenborg, Pernille Lindemann, Tommy Akselsen, Knut Johan Bere, Tobias Torrissen, Kristrun Gudnadottir, Synne Sofie Stangeland, Brede Barkenes, Armin Triendl, Tim Gfeller, Jukka Leino, William Farstad Olsen.

Det er noen som jeg må takke mye. 



Jeg har også lyst til å takke gode kollegaer på Oslo Nye Høgskole. Først av alt vil jeg takke Randi Bjøntegaard som ville ha meg til å revidere og undervise i kognitiv psykologi sammen med Johanna Katarina Blomster Lyshol. Senere var det Andreas Berg Krosby og Pernille Krog Torp som overtok og ble min nærmeste ledere. Tusen takk for at dere begge alle har vært så støttende og tålmodige med meg. Dette gjorde at jeg hadde det helt topp hos dere. Jeg må også takke Axel Davies Vittersø og Sofie Sagfossen for at jeg fikk dele kontor med dere. Jeg er også takke Peder M Isager for alle diskusjoner om effektstørrelser og ekvivalenstesting. 

Jeg vil også takke venner og kollegaer ved Høgskolen i Innlandet (avdeling Lillehammer). Her vil jeg spesielt takke Sjur Fortun Øfsteng, Knut Sindre Mølmen, Daniel Hammarström, Heidi og Nicki Almquist for god hjelp og støtte under doktorgraden. 

Jeg må også takke Mor og Far for støtten gjennom hele livet og gjennom doktorgraden. Dere har vært noen fantastiske foreldre, som har introdusert meg for mange idretter. Uten dette hadde jeg aldri fullført doktorgraden i idrettsvitenskap. 

Jeg har 


Jeg vil også takke min Twitter/X venner for å ha hjulpet meg gjennom doktorgraden og besvart metode og statististikk spørsmål som jeg har lurt på. Jeg vil her spesielt trekke frem Isabella G. Ghement, Dale Barr, Dan Quintana, Gavin Simpson, Cameron Patrick, Solomon Kurz, James Steele, Peder M Isager, Stephen Wild, Brad McKay, Matthew Kay, Brenton Wiernik, Joseph Bulbulia og Daniël Lakens. Det tok noen år ut i doktorgraden før jeg turte å stille "dumme" spørsmål, men jeg skjønte at det er sånn de beste lærer. Jeg er glad jeg turte og stille disse "dumme" spørsmålene, og at dere var tålmodige, støttende og at dere tok dere tid til å svare. Tusen takk!

Til slutt må jeg takket min kjære samboer og forlovede Kjersti Mejdel Styrmoe. Vi ble sammen da jeg begynte på doktorgraden, på et tidspunkt der jeg var naiv nok til å tro at doktorgraden skulle bli en enkel oppgave. Du har vært min viktigste støttespiller gjennom disse 4.5 årene, og har ikke bare støttet meg på hjemmebane. Du har faktisk stilt opp på datainnsamling og stått flere dager i fryseboksen for at jeg skulle komme i mål. Jeg vet du er lei av at jeg ligger metode og statistikkbøker overalt i leiligheten, og at jeg alltid har med en ny statistikkbok på ferie hver sommer. Jeg lover at fra nå av skal disse stå i hyllene. D
