\setlength{\parindent}{0pt} % Fjerner innrykk
\setlength{\parskip}{1em}   % Legger til mellomrom mellom avsnittene


This doctoral project involved collaboration between the Norwegian School of Sport Sciences (Department of Physical Performance), the Norwegian Ski Federation, SNØ (the indoor ski hall in Oslo), and various ski academies, senior ski teams, and national ski teams in Norway and Sweden. I am grateful for the opportunity I have been provided to undertake such a project, and I hope I have been able to give back to you with knowledge that is both practically applicable and of theoretical interest to the field of psychology. There are many individuals who have helped me along the way and who deserve a heartfelt thank you.

Romy Frömer: I am happy I asked you how you created the figures in your \textit{eLife} article. This started our collaboration, and you officially became my supervisor in the fall of 2023. I am immensely thankful for your excellent supervision through our weekly Monday meetings and close follow-ups. You have trained me in scientific thinking, open science, and writing. I have been very privileged to have you as my supervisor.

Per Haugen: Thank you for all the discussions and help I received from you during my studies at the Norwegian School of Sport Sciences. Your door has always been open, and you have always been someone I can call if I get stuck—whether it is with research or struggling with a dough that will not rise. You are the most knowledgeable person about alpine skiing, and I greatly appreciate your expertise.

Robert Reid: Thank you for the discussions we have had around the projects—from bachelor's to PhD—and for supporting and anchoring the project in the Norwegian Ski Federation's research and development strategy.

Thomas Losnegard: Thank you for your genuine interest in motor learning and for supporting the project. Thank you for your valuable advice and tips during my doctoral studies.

Matthias Gilgien: Thank you for your contribution during data collection and for being my supervisor.

In addition, many people have worked around the clock to help me complete the data collection for the projects. Several have set their alarms for 2 AM for weeks on end to get to the ski hall early enough to prepare for the skiers.

Rune Velta, Petter Husevåg Jølstad, Jasper Bates, Andreas Lodden, Kasper Sjøstrand, Kjersti Mejdell Styrmoe, Andreas Lodden, Johan Ansnes, Magne Lund-Hansen, Einar Witteveen, Allan Henriksen, Tom Knudsen, Victoria S Placth, Sara Stensø, Ørjan Lydersen, Eirik Knutsen, Haakon Staff, Live Luteberget, Sindre Lindstad Hoholm, Aurora Haug, Tor-Magnus Blakstad Cappelen, Katrine Groth Eggar, Mai Sissel Linløkkel, Simen Leithe Tajet, Erland Vedeler Stubbe, Erland Hoff Thommasen, Jeremy Phyffer, Tim Gfeller, Andreas Kollenborg, Pernille Lindman, Tommy Akselsen, Knut Johan Bere, Tobias Torrissen, Kristrun Gudnadottir, Synne Sofie Stangeland, Brede Barkenes, Armin Triendl, Jukka Leino and William Farstad Olsen.

Thank you to the students and colleagues at the Norwegian School of Sport Sciences. Special thanks to Simen Leithe Tajet for being a great discussion partner since we started planning the reinforcement learning article and to Lars Martin Tingelstad for discussions about and methods and statistics. I also thank Petter Husevåg Jølstad for processing the data in MATLAB for Paper II. Finally, a big thank you to Jørgen Jensen for all discussions about science and for teaching me how to write.

I would also like to thank my friends in the online science WhatsApp group: James Steele, Christopher Matthews, Matthew Peter Shaw, and Liam Satchell. I have never been part of a group with such varied backgrounds, and I have learned much from our scientific discussions. James Steele, thank you for pushing me to read Paul Meehl’s papers.

My colleagues at Oslo New University College deserve thanks as well. First, Randi Bjøntegaard, thank you for inviting me to revise and teach cognitive psychology alongside Johanna Blomster Lyshol. Later, Andreas Berg Krosby and Pernille Krog Torp became my closest leaders—thank you for your support and patience, which made my time with you fantastic. I must also thank Axel Davies Vittersø and Sofie Sagfossen for sharing an office with me and Peder Mortvedt Isager for our discussions about effect sizes and equivalence testing.

I am also grateful to my friends and colleagues at Inland Norway University of Applied Sciences (Lillehammer Department). Special thanks to Sjur Fortun Øfsteng, Knut Sindre Mølmen, Daniel Hammarström, Heidi Bråten Richenberg, and Nicki Almquist for their great discussions and support during my PhD.

I would like to thank SNØ and Igloo Innovation for their excellent collaboration. Without your goodwill, this project would never have been realized.

I would also like to give a special thanks to some researchers in motor skill learning. Keith Lohse, thank you for your help and for allowing me to take the longitudinal data analysis course in R with your graduate students in University of Utah in the spring of 2021. This course inspired me to delve deeper into multilevel statistics. I would also like to thank Sarah Taylor for our discussions about the design of the contextual interference study. I also want to extend my gratitude to Elizabeth Ligon Bjork and Robert A. Bjork for answering all my questions about metacognition and contextual interference and for connecting me with Timothy Lee. Timothy Lee, thank you for your insights and discussions regarding the article on contextual interference.

I extend my heartfelt gratitude to my Twitter/X contacts who supported me throughout my doctoral project by participating in discussions and answering my questions about methods and statistics. Special thanks to Isabella R. Ghement, Gavin Simpson, Cameron Patrick, Solomon Kurz, James Steele, Dan Quintana, Peder M. Isager, Stephen Wild, Brad McKay, Matthew Kay, Brenton Wiernik, Jordan Nafa, Rajiv Ranganathan, Joseph Bulbulia, Dale Barr, and Daniël Lakens. It took a few years into my PhD before I felt comfortable asking "silly" questions, but I realized this is how the best learning happens. I deeply appreciate your patience, support, and the time you took to respond. Thank you very much.

I also want to thank Christian Mitter, Steve Skavik, and Knut Johan Bere for great discussions about the reinforcement learning article specifically and alpine skiing in general.

Thank you to my friends and family, especially my mother and father, who have been my most important supporters throughout my life. If it hadn't been for you introducing me to so many sports when I was young, I probably would not be involved in sports today, and this doctoral thesis would never have become a reality. Thank you for always being there for me.

Finally, I must thank my dear partner and fiancée, Kjersti Mejdell Styrmoe, whom I met just a few weeks after starting my PhD. You have been with me throughout this entire journey, serving as my most important supporter. You are the most understanding, patient, and caring person I have ever met. Not only have you supported me after long days at work, but you have also braved the "freezer" at SNØ to help with data collection. I promise that from now on, you will not find statistics books scattered around the house—from the kitchen to the bedroom. These books will now have a proper home.



\setlength{\parindent}{15pt} % Standard innrykk (kan justeres etter behov)
\setlength{\parskip}{0pt}    % Standard avstand mellom avsnitt (kan justeres etter behov)